\documentclass[a4paper,12pt]{scrartcl}

% Title page
\usepackage{authblk}

% General style
\usepackage{fontspec}
\setmainfont[BoldFont = GenBasB.ttf]{Gentium Plus}
\setsansfont{DejaVu Sans}
\usepackage{newunicodechar}
\newunicodechar{→}{\fontspec{Gentium Plus}→}

% Formulas
\usepackage{amsmath}

% Internal references
\usepackage[titletoc,title]{appendix}
\usepackage{hyperref}
\usepackage[capitalize]{cleveref}
\Crefname{appsec}{Appendix}{Appendices}
\crefname{appsec}{appendix}{appendices}

% Draft style
\usepackage{setspace}
\setstretch{1.5}
\usepackage[top=2.5cm, bottom=2.5cm, left=2.2cm, right=3.5cm]{geometry}
\usepackage{lineno} 

% Figures
\usepackage{subcaption}
\usepackage[font=small,labelfont=it]{caption}
\usepackage{graphicx}

% Code inclusion with syntax highlighting
\usepackage{minted}
\setminted{fontsize=\footnotesize}
\usepackage{markdown}

% Bibliography
\usepackage[backend=biber,
            bibstyle=biblatex-sp-unified,
            citestyle=sp-authoryear-comp,
            maxcitenames=2,url=false,
            maxbibnames=99]{biblatex}
\renewcommand*{\bibfont}{\small}
\addbibresource{everything.bib}

\renewbibmacro*{doi+eprint+url}{%
  \printfield{doi}%
  \newunit\newblock%
  \iftoggle{bbx:eprint}{%
    \usebibmacro{eprint}%
  }{}%
  \newunit\newblock%
  \iffieldundef{doi}{%
    \usebibmacro{url+urldate}}%
  {}%
}


\title{An Agent-Based Simulation for emergent culture areas}
\date{\today}
\author[1]{Gereon A. Kaiping}
%\author[1]{Peter Ranacher}
%\author[2]{David Inman}
\affil[1]{Geographic Information Science Center, Universität Zürich, CH}
%\affil[2]{Institut für vergleichende Sprachwissenschaft, Universität Zürich, CH}

\begin{document}
\maketitle
\section{Introduction}
Agent-Based Models (ABM) \parencite{} can be a useful tool for the investigation
of complex systems not otherwise accessible to the researcher \parencite{}. One
area of application for agent-based simulations is in theoretical biology and
adjacent fielnds, where experiments cannot be conducted due to the scales
involved. ABMs have been fruitfully applied to study the evolution of
cooperation \parencite{}, culture \parencite{}. In the wider historical
sciences, ABMs can help archeologists and paleontolgists interpret scarce data
\parencite{cegielski2016rethinking,barcelo2016simulating}.

In historical linguistics, simulation and other highly computational methods
have only become part of the toolbox, starting with the availability of big
cross-linguistic databases and methods taken over from bioinformatics
\parencite{}. In this field, Agent-Based Models have been applied to study very
specific linguistic changes
\cite{bloem2015agentbased,feltgen2017modeling,grim2011simulating,radulescu2016modelling,vantrijp2016evolution}
or extremely abstract, focussing on evolutionary dynamics
\cite{arita1998evolution,castello2013agentbased,javarone2014competitive}.
However, the statistical methodology applied eg. by studies of language
phylogeny largely fall into two camps: Many models are taken over from
biological evolution and insufficiently grounded in linguistic reality. Others
are justified by mathematical simplicity, but have not been tested for
linguistic realism either. ABMs, as bottom-up simulations which other more
abstract statistical methods can be validated against, will help bridge the gap
between small-scale linguistic processes and large-scale statistical analyses.

In this article, we present an Agent-Based Model in which areas of coherent
culture or language arise from low-level demographic and migratory processes.
The model provides a first step towards understanding the patterns of language
diversification, spread, and extinction. These processes are important factors
in the large-scale dynamics of language evolution. They directly inform the
shape of tree priors in Bayesian phylogenetics, which are currently heavily
relying on theoretical biology (for coalescent and birth/death priors) or
reduced to maximum mathematical simplicity (in the case of uniform tree priors)
and have some influence on inferred tree topologies and dates
\cite{rama2018three}.

The model focusses on hunter-gatherer cultural spread. While horticulture and
agriculture have been developed independently in many parts of the world
\parencite{} and thus directly or indirectly influenced most of the worlds'
cultures, this development is somewhat recent in the scope of human history.
More importantly for the purposes of this initial model, agriculture would
vastly increases the complexity of the model, because it would add a feedback
loop between culture and carrying capacity. This additional complexity will be
necessary for a full model of language dispersal, but in order to piece apart
the various influences, an iterative approach will be necessary.

Two further sharp simplifications our model has to resort to for this first step
are as follows. No language contact (additional complexity, and no reliable
quantitative data), no complex model of culture (not focus of the analysis, no
reliable quantitative results either).

As a summary, we construct a demographic migration model which includes culture.
The model is designed with extension to more concrete research questions in
mind. In the current, first stage, the purpose of the model is to investigate
how languages disperse and split.

The structure of this article is as follows. In \cref{s:earlier}, we provide an
overview over existing models that were considered in the construction of the
present model, and which of their elements were included into our model.
\Cref{s:odd} provides the model description according to the revised ODD
protocol \parencite{grimm2006standard,grimm2010odd}, generated using literate
programming techniques \parencite{knuth1984literate} from the Rust \parencite{}
source code of the simulation. We summarize our parameter choices in
\cref{s:parameters}. In \cref{s:results} we present the results from the various
runs of the simulation with the aforementioned parameter settings. We discuss
our results in \cref{s:discussion}. The paper finishes with conclusions in
\cref{s:conclusions}.

\section{Earlier Models of hunter-gatherer cultural dispersal}
\label{s:earlier}

To construct a demographic migration model with culture, for the purpose of the
project we set out here, we need the following ingedients.
\begin{enumerate}
\item A representation of culture, which is at the least able to undergo neutral
  evolution (showing heritability and random variation, not necessarily
  fitness) and which in addition allows horizontal transfer of cultural traits
  other than from a parent to a child population.
\item Agents that carry cultural traits and are located in geographical space,
  which has differing ecological features
\item A system that drives the demographics of the agents in time and space
\item A way for culture and population dynamics to interact in a way that can
  create create distinct cultural areas instead of a vast cultural cline or
  dialect continuum.
\end{enumerate}

In the following subsections, we will consider each of these elements
separately.

\subsection{Modeling culture}
\label{s:culture}
In order to display evolutionary dynamics, a complex system must at its minimum
have properties that can be passed on from ancestors to descendants, which
undergo variation throughout time. For evolution in the classical non-neutral
sense, it is furthermore necessary for the properties to be adaptive, i.\,e.
have an influence on survival and number of descendants. The discussion to what
extent culture in general is adaptive vs. largely neutral is ongoing
\parencite{}. There certainly are cultural domains that show adaptation and have
effects on survivability, such as climate-appropriate clothing and shelter
\parencite{} or methods of subsistence from gathering to agriculture
\parencite{}. 

Of these, agriculture is explicitly outside the scope of the
current model, because it introduces a feedback loop between culture and
carrying capacity which would mask other effects. A vast part of the
history of humanity went about without agriculture \parencite{}. But once it
arose, around 5000 to 10000 years ago independently in different parts of the
world \parencite{}, it presumably became a major driver of cultural spreads
\parencite{farming/language dispersal hypothesis} and as such it will be useful
to add such effects to a later iteration of this model.

Certain cultural traits are obviously important for migration processes. Due to
lack of comparable data on such processes on the family level, we will take an
ecology-driven probability density to govern knowledge of neighboring patches,
and assume this captures the direct interactions between the cultural process
and the demographic process. We describe the motivation of this approximation
below in \cref{s:geography}.

Beyond traits directly interacting with the ecological niche of a society, there
is significant literature on the interaction of some very specific cultural
traits (eg. \textcite{rusch2014evolutionary} on altruism and inter-group
conflict, \textcite{watts2016ritual} on stratified societies and human
sacrifice), but very few models that consider specific meaniningful cultural
traits in agent-based, broad, and geographically expansive models. The closest
to our goals here may be \textcite{hofstede2012cultural}. They locate their
agents on \textcite{hofstede2001culture}'s cultural dimensions (which are
themselves not beyond harsh criticism
\parencite{mcsweeney2002hofstede,baskerville2003hofstede}) in order to model
negotiations between agents of different cultural backgrounds.
Diverging
% I should not use the word ‘crackpot’ here.
hypotheses notwithstanding \parencite{tone-humidity,elevation-ejectives,etc}, it
seems that a vast number of cultural traits are not directly involved in niche
adaptation. This does not mean individual traits do not exhibit non-neutral
evolutionary dynamics, as shown eg. by \textcite{kandler2013non}. But a neutral
model for cultural change appears appropriate nonetheless.

Neutral abstract models for culture have in the past been considered for various
purposes \parencite{komarova2001evolutionary}. In such models, culture tends to
be modeled as a binary vector
\parencite{fogarty2017driving,pascual2020epistasis}. The number of dimension $M$
of this culture vector range between $M=6$ \parencite{pascual2020epistasis} and
$M=10$ \parencite{delcastillo2013modeling}, with empirical reasons cited for $ $
\parencite{} and theoretical reasons or $ $ \parencite{} or $M>N$ for the
(effective) population size $N$ \parencite{fogarty2017driving}. Vectors are
compared using the Hamming distance (after \textcite{}, measuring the number of
mismatches between the two vectors) in most cases
\parencite{fogarty2017driving,pascual2020epistasis}.

Other options exist. For instance,
\textcite{barcelo2014social,barcelo2015simulating} use integer vectors with
values between 1 and 6 reflecting importance, and ‘a multidimensional weighted
Euclidean distance based on the extension of the cosine similarity measure for
vectors’ (2014), with weights that ‘roughly imitate[] the results of a factorial
analysis of individual beliefs’ (2014). It has not been shown that this approach
to modeling culture improves realism or interpretability, so we take a neutral
evolution model with binary vectors and an unweighted Hamming distance.

\subsection{Hunter-gatherer geography}
\label{s:geography}
Cultural traits can be attached on various levels – to individuals, bands,
territorial groups, or nation states, depending on the desired level of
abstraction. For our model, we want to see whether the components we consider
are sufficient to let culturally similar regions, with sharp boundaries, arise
from smaller scale interactions. For hunter-gatherer societies, the family has
been suggested as the decision-making agent driving migration and demographics
\textcite{}, so it makes sense to consider families the agents of our model and
attach cultural traits on the family level. Families tend to be socially and
geographically coheherent \textcite{}, but share their range with other
families, which makes co-locality a reasonable factor for social interactions
between agents.

It has been shown that language families spread easier east-west than
north-south, due to climate varying much more with latitude than with longitude
\parencite{}. Local biome can also play a role in the dispersal of a language
family, as argued eg. by \textcite{grollemund2015bantu,ehret2015bantu}. There is
not much literature on explicit models of cultural traits interacting with
ecological niche in the context of migration. Eco-Cultural Niche Modeling
\parencite[ECNM]{banks2006ecocultural} has been a useful tool for a variety of
archaeological questions
\cite{banks2008human,banks2013ecological,dalpoimguedes2014modeling,kondo2018ecological,walker2019persistence}.
A well-established tool for ECNM is the maximum entropy approach implemented in
the MaxEnt software package
\cite{phillips2006maximum,phillips2008modeling,maxenttutorial}. MaxEnt
generates, based on a set of points and environmental parameters, a maximum
entropy probabilistic model for predicting the locations of these points. The
resulting probability surface has been used to construct likely migration
pathways of prehistoric populations \cite{kondo2018ecological}.

This suggests in converse that well-constructed cost-surface paths can be a tool
to generate an approximation of a probability distribution representing niche.
We construct a cost function that describes the effort moving from one
settlement location to another, in terms of raw energy expenditure and missed
foraging opportunities due to time spent on the trail. For land-based travel, we
follow \textcite{wood2006energetically}, which provides a metabolics-based cost
that improve upon earlier heuristics such as \textcite{tobler1993three}'s. For
waterways, these costs are supplemented by \textcite{livingood2012no}, and land
cover that might increase the effort of movement is taken from
\cite{white2012geospatial}. The cost is used in two ways: It serves as a penalty
when a family considers moving to a better patch, which means that a neighboring
patch needs to be better than the current patch by a margin. (It thus fulfills a
similar role to the threshold of evidence $c$ by
\textcite{crema2014simulation}.) In addition, a family will only move to a place
they know about, and such knowledge can only be acquired by a human traversing
the distance. This exploration is not modeled explicitly, but it gives an
intuitive mapping from the cost surface to a probability surface: Instead of
visiting $q$ patches with some arbitrary travel cost, an explorer can visit one
patch with the $q$-fold effort to reach it and will still cross through at least
one patch at the original cost. So we take anything that can be reached with an
expenditure $E$ of one day's worth of resources (2263 kcal, according to
\textcite{pontzer2012huntergatherer}---\textcite{smith2003assessment} cites a
similar number of 2390 kcal) to be known ($p=1$), and for other locations with a
cost-to-reach of $E$ accordingly we take $p(E) = \frac{1}{2 + E / {2263
    \text{ kcal}}}$ for all patches that can be reached within XXX days
\parencite{}, and $p(E) = 0$ otherwise.

\subsection{Hunter-gatherer demographics}
\label{s:demographics}

\textcite{hamilton2018stochastic}: Very simple model. Interesting extinction
mechanics, which I would love to include, but don't know how yet.

\textcite{crema2014simulation}: Fission-fusion dynamics, because that drives
migration and cooperation. We use a slightly reduced model of their decision
process.

\subsection{Interaction between culture and demographics}
\label{s:interaction}

Culture drives cooperation, cooperation pushes cultural assimilation. Following
\parencite{barcelo2014social,barcelo2015simulating}, but deviating from their
implementation (see \cref{s:geography} for some reasons why we can), we take this
as the core interation feedback loop.

\section{ODD model description}
\label{s:odd}

\inputminted{rust}{supplement/dispersal_model_rust/src/main.rs}

\section{Parameter choices}
\label{s:parameters}
\section{Results}
\label{s:results}
\section{Discussion}
\label{s:discussion}
\section{Conclusions}
\label{s:conclusions}

\printbibliography
\end{document}

% Local Variables:
% TeX-engine: luatex
% End:
