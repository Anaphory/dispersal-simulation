\documentclass[a4paper,12pt]{scrartcl}

\usepackage{authblk}

% Draft style
% \usepackage{setspace}
% \setstretch{1.5}
% \usepackage[top=2.5cm, bottom=2.5cm, left=2.2cm, right=3.5cm]{geometry}
% \usepackage{lineno} 

\usepackage{hyperref}
\usepackage[capitalize]{cleveref}

% Bibliography
\usepackage[backend=biber,
            bibstyle=biblatex-sp-unified,
            citestyle=sp-authoryear-comp,
            maxcitenames=2,url=false,
            maxbibnames=99]{biblatex}
\addbibresource[location=remote]{/home/gereon/Downloads/MyLibrary.bib}
\addbibresource[location=remote]{/home/gereon/Downloads/VICI.bib}
\renewcommand*{\bibfont}{\small}

\renewbibmacro*{doi+eprint+url}{%
    \printfield{doi}%
    \newunit\newblock%
    \iftoggle{bbx:eprint}{%
        \usebibmacro{eprint}%
    }{}%
    \newunit\newblock%
    \iffieldundef{doi}{%
        \usebibmacro{url+urldate}}%
        {}%
      }


\begin{document}
The basic shape of the model is remotely inspired by
\textcite{gavin2017processbased}.

A hexagonal grid is put over the Americas, with target area 450\,000\,000~m$^2$.
The largest connected component of the Americas is covered by XXX cells under
this assumption.
This puts the radius of each cell at about 12~km, which is commensurate with
both the purposes of describing language diversity, as described by
\textcite{gavin2017processbased}, and with modeling hunter-gatherer migration
patterns, where movement it expected to be in the order of magnitude of two
foraging radii \parencite{grove2009hunter}, for which \textcite[Table
7.13]{binford2001constructing} gives a mean of 15.94~km.

We take the expected population density TERMD2 \parencite[Formula
6.15]{binford2001constructing} as carrying capacity for each cell.
\Author{binford2001constructing} derives the formula for carrying capacity
from first principles (duration of the growth season, food conversion rates,
etc.) and validates it against his data of 390 worldwide hunter-gatherer
socities with sufficient ethnographic description to be comparable. A coarser
estimate using similar principles is given by \textcite{tallavaara2015human}.

At a later stage, we will use paleoclimate data to compute those carrying
capacities. \Author{binford2001constructing} focuses on making data useful for
archeologists and as such derives the carrying capacity from quantities that can
likely be estimated at deep time scales far before the begin of meteorology and
quantitative science.


We assume that migration might be influenced by seasonality in parts of the world. We will calibrate
the model for time steps of 1/2 year, so we can at a later stage add two
different, seasonal, carrying capacity computations for each cell.
This choice is likely not necessary for tropical regions, but it should make a
significant difference for non-tropical climates.

A challenge for this approach is that the TERMD2 data is limited to
hunter-gatherer populations, and does not even include a specific consideration
of food storage differences yet. While some extension to food storage will be
useful for the present analysis, horticulture and agriculture are outside the
scope for now. For the time being, we hope that regions where agriculture plays a
role will show up as discrepancies between expected and observed language
diversity. A follow-up study can then investigate whether the model can be
extended to include the invention and spread of agriculture.

For the migration and spread of the populations, we adapt the model of
\textcite[scenario 4]{crema2015modelling}, which is an expansion of his earlier
\citeyear{crema2014simulation} model, adding resource depletion.
(\textcite{gargano2017largescale} construct an extremely abstract model looking at a
similar phenomenon. While they claim that the model shows effect of ‘conflict in
pre-agricultural groups’, they say “This produces a reduction of the total human
populations in $\alpha$ and $\beta$, which we interpret as the outcome of increased mortality
due to competition and/or conflict over local resources.”, so the effects of
conflict cannot be lifted from there model separately.)

While it is generally
assumed that populations form a hierarchical structure of individuals forming
family groups inside of bands as part of aggregated groups in larger populations
(see \textcite{hamilton2007complex} for a quantitative study),
\textcite{bird2019variability} argue that bands are much more flexible,
composing themselves out of unrelated people from different dialect groups.

Following \textcite{delcastillo2013modeling}, we take the former stance and
expand the model so far as to even assume that the household or family is the
fundamental agent that makes decisions, forages, and moves. We consider their
model as reference for including cultural traits and divergence in this
population dynamics model.

We adapt that model to the hexagonal grid and differential carrying capacity,
and attempt to choose parameters to reflect real human migration, birth and
death patterns following eg. \textcite{hamilton2007complex} – We do not assume a
maximum group size, because we will add a social structure that will give rise
to emerging language populations in the next steps.

Following \textcite{hamilton2018stochastic}, we assume that populations need a
minimum size to viable and that they immediately die out if they are below that
threshold. We also incorporate their notion of catastrophic events occurring at
a constant rate killing a uniformly distributed fraction of the population.
Using the social network structure, we will in a later stage be able to import
models from epidemiology to give these catastrophic events a more realistic
impact on the population than a random selection would.

If the model is realistic, it should tend towards minimizing the ‘tension’ of
language boundaries as observed eg. by \textcite{burridge2017spatial}, and
towards realistic rates of language diversification as inferred in Bayesian
phylogenetic studies.

\printbibliography
\end{document}
