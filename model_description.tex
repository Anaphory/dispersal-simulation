\documentclass[a4paper,12pt]{scrartcl}

% General style
\usepackage{fontspec}
\setmainfont[BoldFont = GenBasB.ttf]{Gentium Plus}
\setsansfont{DejaVu Sans}
\usepackage{newunicodechar}
\newunicodechar{→}{\fontspec{Gentium Plus}→}

% Internal references
\usepackage[titletoc,title]{appendix}
\usepackage{hyperref}
\usepackage[capitalize]{cleveref}
\Crefname{appsec}{Appendix}{Appendices}
\crefname{appsec}{appendix}{appendices}

\usepackage{authblk}

% Draft style
\usepackage{setspace}
\setstretch{1.5}
\usepackage[top=2.5cm, bottom=2.5cm, left=2.2cm, right=3.5cm]{geometry}
\usepackage{lineno} 

% Figures
\usepackage{subcaption}
\usepackage[font=small,labelfont=it]{caption}
\usepackage{graphicx}

% Code inclusion with syntax highlighting
\usepackage{minted}
\setminted{fontsize=\footnotesize}

% Bibliography
\usepackage[backend=biber,
            bibstyle=biblatex-sp-unified,
            citestyle=sp-authoryear-comp,
            maxcitenames=2,url=false,
            maxbibnames=99]{biblatex}
\renewcommand*{\bibfont}{\small}
\addbibresource{everything.bib}

\renewbibmacro*{doi+eprint+url}{%
  \printfield{doi}%
  \newunit\newblock%
  \iftoggle{bbx:eprint}{%
    \usebibmacro{eprint}%
  }{}%
  \newunit\newblock%
  \iffieldundef{doi}{%
    \usebibmacro{url+urldate}}%
  {}%
}


\title{An Agent-Based Simulation for emergent culture areas}
\date{\today}
\author[1]{Gereon A. Kaiping}
\author[1]{Peter Ranacher}
\author[2]{David Inman}
\affil[1]{Geographic Information Science Center, Universität Zürich, CH}
\affil[1]{Institut für vergleichende Sprachwissenschaft, Universität Zürich, CH}

\begin{document}
\maketitle
\section{Introduction}
Agent-Based Models (ABM) \parencite{} can be a useful tool for the investigation
of complex systems not otherwise accessible to the researcher \parencite{}. One
area of application for agent-based simulations is in theoretical biology and
adjacent fielnds, where experiments cannot be conducted due to the scales
involved. ABMs have been fruitfully applied to study the evolution of
cooperation \parencite{}, culture \parencite{}. In the wider historical
sciences, ABMs can help archeologists interpret scarce data
\parencite{cegielski2016rethinking}.

In historical linguistics, simulation and other highly computational methods
have only become part of the toolbox, starting with the availability of big
cross-linguistic databases and methods taken over from bioinformatics
\parencite{}. In this field, Agent-Based Models have been applied to study very
specific linguistic changes
\cite{bloem2015agentbased,feltgen2017modeling,grim2011simulating,radulescu2016modelling,vantrijp2016evolution}
or extremely abstract, focussing on evolutionary dynamics
\cite{arita1998evolution,castello2013agentbased,javarone2014competitive}.
However, the statistical methodology applied eg. by studies of language
phylogeny largely fall into two camps: Many models are taken over from
biological evolution and insufficiently grounded in linguistic reality. Others
are justified by mathematical simplicity, but have not been tested for
linguistic realism either. ABMs, as bottom-up simulations which other more
abstract statistical methods can be validated against, will help bridge the gap
between small-scale linguistic processes and large-scale statistical analyses.

In this article, we present an Agent-Based Model in which areas of coherent
culture or language arise from low-level demographic and migratory processes.
The model provides a first step towards understanding the patterns of language
diversification, spread, and extinction. These processes are important factors
in the large-scale dynamics of language evolution. They directly inform the
shape of tree priors in Bayesian phylogenetics, which are currently heavily
relying on theoretical biology (for coalescent and birth/death priors) or
reduced to maximum mathematical simplicity (in the case of uniform tree priors)
and have some influence on inferred tree topologies and dates
\cite{rama2018three}.

The model focusses on hunter-gatherer cultural spread. While horticulture and
agriculture have been developed independently in many parts of the world
\parencite{} and thus directly or indirectly influenced most of the worlds'
cultures, this development is somewhat recent in the scope of human history.
More importantly for the purposes of this initial model, agriculture would
vastly increases the complexity of the model, because it would add a feedback
loop between culture and carrying capacity. This additional complexity will be
necessary for a full model of language dispersal, but in order to piece apart
the various influences, an iterative approach will be necessary.

Two further sharp simplifications our model has to resort to for this first step
are as follows. No language contact (additional complexity, and no reliable
quantitative data), no complex model of culture (not focus of the analysis, no
reliable quantitative results either).

As a summary, we construct a demographic migration model which includes culture.
The model is designed with extension to more concrete research questions in
mind. In the current, first stage, the purpose of the model is to investigate
how languages disperse and split.

The structure of this article is as follows. In \cref{s:earlier}, we provide an
overview over existing models that were considered in the construction of the
present model, and which of their elements were included into our model.
\Cref{s:odd} provides the model description according to the revised ODD
protocol \parencite{grimm2006standard,grimm2010odd}, generated using literate
programming techniques \parencite{knuth1984literate} from the Rust \parencite{}
source code of the simulation. We summarize our parameter choices in
\cref{s:parameters}. In \cref{s:results} we present the results from the various
runs of the simulation with the aforementioned parameter settings. We discuss
our results in \cref{s:discussion}. The paper finishes with conclusions in
\cref{s:conclusions}.

\section{Earlier Models of hunter-gatherer cultural dispersal}
\label{s:earlier}

To construct a demographic migration model with culture, for the purpose of the
project we set out here, we need the following ingedients.
\begin{enumerate}
\item A representation of culture, which is at the least able to undergo neutral
  evolution (showing heritability and random variation, not necessarily
  fitness) and which in addition allows horizontal transfer of cultural traits
  other than from a parent to a child population.
\item Agents that carry cultural traits and are located in a geographical space
\item A system that drives the demographics of the agents in time and space
\item A way for culture and population dynamics to interact in a way that can
  create create distinct cultural areas instead of a vast cultural cline or
  dialect continuum.
\end{enumerate}

In the following subsections, we will consider each of these elements
separately.

\subsection{Modeling culture}
\label{s:culture}

\subsection{Hunter-gatherer geography}
\label{s:agents}

\subsection{Hunter-gatherer demographics}
\label{s:demographics}
\section{Overview, Design concepts, and Details}
\label{s:odd}

\inputminted{rust}{supplement/dispersal_model_rust/src/main.rs}
\section{Parameter choices}
\label{s:parameters}
\section{Results}
\label{s:results}
\section{Discussion}
\label{s:discussion}
\section{Conclusions}
\label{s:conclusions}

\printbibliography
\end{document}

% Local Variables:
% TeX-engine: luatex
% End:
