\documentclass[a4paper,12pt]{scrartcl}

% Title page
\usepackage{authblk}

% General style
\usepackage{fontspec}
\setmainfont[BoldFont = GenBasB.ttf]{Gentium Plus}
\setsansfont{DejaVu Sans}
\usepackage{newunicodechar}
\newunicodechar{→}{\fontspec{Gentium Plus}→}

% Formulas
\usepackage{amsmath}

% Internal references
\usepackage[titletoc,title]{appendix}
\usepackage{hyperref}
\usepackage[capitalize]{cleveref}
\Crefname{appsec}{Appendix}{Appendices}
\crefname{appsec}{appendix}{appendices}

% Draft style
\usepackage{setspace}
\setstretch{1.5}
\usepackage[top=2.5cm, bottom=2.5cm, left=2.2cm, right=3.5cm]{geometry}
\usepackage{lineno} 

% Figures
\usepackage{subcaption}
\usepackage[font=small,labelfont=it]{caption}
\usepackage{graphicx}

% Bibliography
\usepackage[backend=biber,
            bibstyle=biblatex-sp-unified,
            citestyle=sp-authoryear-comp,
            maxcitenames=2,url=false,
            maxbibnames=99]{biblatex}
\renewcommand*{\bibfont}{\small}
\addbibresource{everything.bib}

\renewbibmacro*{doi+eprint+url}{%
  \printfield{doi}%
  \newunit\newblock%
  \iftoggle{bbx:eprint}{%
    \usebibmacro{eprint}%
  }{}%
  \newunit\newblock%
  \iffieldundef{doi}{%
    \usebibmacro{url+urldate}}%
  {}%
}


% Code inclusion with syntax highlighting
\usepackage{minted}
\setminted{fontsize=\tiny,baselinestretch=0.9}
\usepackage{markdown}

\title{An Agent-Based Simulation for emergent culture areas}
\date{\today}
\author[1,2]{Gereon A. Kaiping}
%\author[1,2]{Peter Ranacher}
%\author[3]{David Inman}
\affil[1]{Geographic Information Science Center, Universität Zürich, CH}
\affil[2]{NCCR Evolving Language}
%\affil[2]{Institut für vergleichende Sprachwissenschaft, Universität Zürich, CH}

\begin{document}
\maketitle
\section{Introduction}
Global cultural diversity provides us with a unique window into the past, and
“any reduction of language diversity diminishes the adaptational strength of our
species because it lowers the pool of knowledge from which we can draw”
(Bernard, 1992:82). %Says Daria, my Bachelor student. So I got at least
                    %something out of her thesis ;)
Of particular interest are thus remnants of language diversity before the
wide-spread colonization of the world by European nations. Several recent
studies
\parencite{gavin2017processbased,pachecocoelho2019drivers,hua2019ecological}
consider ecological factors for language diversity. These studies so far stay
largely on the phenomenological level, estimating extant language diversity from
various ecological predictors and finding areas of surprisingly high or low
language diversity. The processes that drive diversity on these scales, on the
basis of small-scale human interactions, however, are not understood.

Agent-Based Models (ABM) \parencite{} can be a useful tool for the investigation
of complex systems not otherwise accessible to the researcher \parencite{}. One
area of application for agent-based simulations is in theoretical biology and
adjacent fields, where experiments cannot be conducted due to the scales
involved. ABMs have been fruitfully applied to study the evolution of
cooperation \parencite{}, culture \parencite{}. In the wider historical
sciences, ABMs can help archeologists and paleontolgists interpret scarce data
\parencite{cegielski2016rethinking,barcelo2016simulating}.

In historical linguistics, simulation and other highly computational methods
have only recently become part of the toolbox, starting with the availability of big
cross-linguistic databases and methods taken from bioinformatics
\parencite{}. In this field, Agent-Based Models have been applied to study very
specific linguistic changes
\cite{bloem2015agentbased,feltgen2017modeling,grim2011simulating,radulescu2016modelling,vantrijp2016evolution}
or extremely abstract, focussing on evolutionary dynamics
\cite{arita1998evolution,castello2013agentbased,javarone2014competitive}.
However, the statistical methodology applied eg.\ by studies of language
phylogeny largely fall into two camps: Many models are taken over from
biological evolution and insufficiently grounded in linguistic reality. Others
are justified by mathematical simplicity, but have not been tested for
linguistic realism either. ABMs, as bottom-up simulations which other more
abstract statistical methods can be validated against, will help bridge the gap
between small-scale linguistic processes and large-scale statistical analyses.

In this article, we present an Agent-Based Model in which the spatial
distribution of languages on the continent scale, in the shape of realistically
sized areas with internally language, arise from low-level demographic and
migratory processes. 
While various models combining two of the relevant aspects exist and a small
number of models combining three of the aspects, we incorporates all four of
geography, culture, migration, and demographics in a bottom-up model.
In particular, our model permits a dynamic equilibrium, where the entire
geographic space is populated, but migration and language change continue to
form the linguistic landscape.
\emph{Previous literature is the one in \cref{s:earlier} – Each needs to be
  classified by which topics it touches on and which are missing.}

As such, our model provides a first step towards understanding the patterns of
language diversification, spread, and extinction. These processes are important
factors in the large-scale dynamics of language evolution, and a computational
model will thus contribute to our understanding of the causes and factors of
linguistic diversity.
In addition, these factors directly inform the
shape of tree priors in Bayesian phylogenetics. This burgeoning field
\parencite{jager,bouckaert,gray,sino-tibetan} grew from the application of
bioinformatics tools to linguistic data. As such, the methodology still heavily
relies on theoretical biology (for coalescent and birth/death priors) or reduced
to maximum mathematical simplicity (in the case of uniform tree priors). Our
model could be used to assess tree priors, which have some influence on inferred
tree topologies and dates \cite{rama2018three}, and generally contribute to the
current push for more linguistic models in language phylogenetics
\parencite{stochasticdollo,dollowithlateraltransfers,neureiter,maurits-rakes,kaiping-tap}.

Being a first step in this direction, the model presented here has three major restrictions.
\begin{enumerate}
\item In the model, we focus on the processes that drive the spatial
  distribution of cultures only for hunter-gatherers. Horticulture and
  agriculture have been developed independently in many parts of the world
  \parencite{} and thus directly or indirectly influenced cultural diversity in
  most parts of the world. However, this development is somewhat recent in the
  scope of human history. More importantly for the purposes of this initial
  model, agriculture would vastly increase the complexity of the model, because
  it would add a feedback loop between culture and carrying capacity. This
  additional complexity will be necessary for a full model of language
  dispersal, but in order to piece apart the various influences, an iterative
  approach will be necessary and we start with the simpler model.
\item When different languages meet, there are only three possible effects:
  Complete assimilation, warfare, or mutual avoidance. The model includes no
  intermediate result, where languages partially assimilate due to
  prestige/warfare/trade/… between families of different culture.
\item no complex model of culture (not focus of the analysis, no reliable
  quantitative results either) – more detailed description and justification of
  culture in the model later.
\end{enumerate}

As a summary, we construct a demographic migration model which includes culture.
The model is designed with extension to more concrete research questions in
mind. In the current, first stage, the purpose of the model is to investigate
how languages disperse and split.

The structure of this article is as follows. The following section provides the model
description according to the revised ODD protocol
\parencite{grimm2006standard,grimm2010odd}, a top-down description protocol for
agent-based models starting with an overview and ending with the details of
individual submodels. We summarize our parameter choices in \cref{s:parameters}.
In \cref{s:results} we present the results from the various runs of the
simulation with the aforementioned parameter settings. We discuss our results in
\cref{s:discussion}. The paper finishes with conclusions in
\cref{s:conclusions}.

\emph{In \cref{s:earlier}, we provide an overview over existing models that were
  considered in the construction of the present model, and which of their
  elements were included into our model. This section well be re-worked into the
  various sections, in particular the ‘submodels’ section, of the ODD, and is
  only a separate section because it was circulated independently for review.}

\markdownInput[citations,fencedCode,headerAttributes]{supplement/dispersal_model_rust/src/lib.rs}

\section{Justifications of model choices}
\label{s:earlier}

\emph{These discussions should be merged into the ODD.}
\subsection{Hunter-gatherer geography}
\label{s:geography}
Cultural traits can be attached on various levels – to individuals, bands,
territorial groups, or nation states, depending on the desired level of
abstraction. For our model, we want to see whether the components we consider
are sufficient to let culturally similar regions, with sharp boundaries, arise
from smaller scale interactions. For hunter-gatherer societies, the family has
been suggested as the decision-making agent driving migration and demographics
\parencite{}, so it makes sense to consider families the agents of our model and
attach cultural traits on the family level. Families tend to be socially and
geographically coheherent \textcite{}, but share their range with other
families, which makes co-locality a reasonable factor for social interactions
between agents.

It has been shown that language families spread easier east-west than
north-south, due to climate varying much more with latitude than with longitude
\parencite{}. Local biome can also play a role in the dispersal of a language
family, as argued eg. by \textcite{grollemund2015bantu,ehret2015bantu}. There is
not much literature on explicit models of cultural traits interacting with
ecological niche in the context of migration. Eco-Cultural Niche Modeling
\parencite[ECNM]{banks2006ecocultural} has been a useful tool for a variety of
archaeological questions
\cite{banks2008human,banks2013ecological,dalpoimguedes2014modeling,kondo2018ecological,walker2019persistence}.
A well-established tool for ECNM is the maximum entropy approach implemented in
the MaxEnt software package
\cite{phillips2006maximum,phillips2008modeling,maxenttutorial}. MaxEnt
generates, based on a set of points and environmental parameters, a maximum
entropy probabilistic model for predicting the locations of these points. The
resulting probability surface has been used to construct likely migration
pathways of prehistoric populations \cite{kondo2018ecological}.

\Textcite{steele2003where} argue for models that take into account not only the
quality (eg. in terms of caloric content) of resources, but also the
accessibility. This suggests in converse that well-constructed cost-surface
paths can be a tool to generate an approximation of a probability distribution
representing niche. We construct a cost function that describes the effort
moving from one settlement location to another, in terms of raw energy
expenditure and missed foraging opportunities due to time spent on the trail.
For land-based travel, we follow \textcite{wood2006energetically}, which
provides a metabolics-based cost that improve upon earlier heuristics such as
\textcite{tobler1993three}'s. For waterways, these costs are supplemented by
\textcite{livingood2012no}, and land cover that might increase the effort of
movement is taken from \cite{white2012geospatial}.

The cost is used in two ways: It serves as a penalty when a family considers
moving to a better patch, which means that a neighboring patch needs to be
better than the current patch by a margin. (It thus fulfills a similar role to
the threshold of evidence $c$ by \textcite{crema2014simulation}.)

In addition, a family will only move to a place they know about, and such
knowledge can only be acquired by a human traversing the distance. This
exploration is not modeled explicitly, but it gives an intuitive mapping from
the cost surface to a probability surface: Instead of visiting $q$ patches with
some arbitrary travel cost, an explorer can visit one patch with the $q$-fold
effort to reach it and will still cross through at least one patch at the
original cost. So we take anything that can be reached with an expenditure $E$
of one day's worth of resources (2263 kcal, according to
\textcite{pontzer2012huntergatherer}---\textcite{smith2003assessment} cites a
similar number of 2390 kcal) to be known ($p=1$), and for other locations with a
cost-to-reach of $E$ accordingly we take $p(E) = \frac{1}{2 + E / {2263 \text{
      kcal}}}$ for all patches that can be reached within XXX days \parencite{},
and $p(E) = 0$ otherwise.

There may be a cost to moving from one habitat to another in the mid-term due to
unfamiliarity not just with the details of the new landscape (“landscape
learning”), but also with the general strategies for dealing with gathering food
in a different habitat or ‘megapatch’ \parencite{kelly2003colonization}.
\Textcite{kelly2003colonization} describes the time to learn a specific
landscape. He does not quote any numbers, but gives impressions between ‘a
couple of years’ and the time from a 12-year olds first accompanying their
father until, presumably, adulthood, so we assume that it might take around 8
years to gain familiarity and full efficiency. A graph by \Textcite[Fig.
4]{freeman2017theory} suggests that under certain circumstances, foraging effort
within a single environment can flip by a factor of roughly 2 due to a bistable
stability landscape for foraging. Due to the unavailability of any other
comparable data, we combine these two very rough estimates into a coarse process
(and later check robustness) as follows.

We assume that moving into a completely new biome will reduce the efficiency of
a family to $0.5$. This has the side-effect to make moving across biome
boundaries more likely for larger groups, who can compensate for the loss in
efficiency by their cooperation. This ties into one of the possible strategies
\textcite{kelly2003colonization} suggests for colonizing new land. Then, in each
year, the efficiency in all biomes decreases by a factor of $\sqrt[30]{0.5}$ to
a minimum of $50\%$, so that after a generation of living entirely outside the
old biome, all specific knowledge is lost, and the family's efficiency in their
current biome increases by $6.25\%$ to a maximum of $100\%$\footnote{An even
  more elegant way would be a yearly loss by $p$ and an increase by $\frac{1}{p}
  -1$, which would naturally balance each other at maximum efficiency $100\%$}, so
that a family reaches full familiarity within 8 years. If a single patch
displays a mix of biomes, a family will adapt randomly in proportion to the
biomes present, thus balancing the adaptation and permitting a gradual
transition between biomes.

\subsection{Hunter-gatherer demographics}
\label{s:demographics}

\textcite{hamilton2018stochastic}: Very simple model. Interesting extinction
mechanics, which I would love to include, but don't know how yet.

\textcite{crema2014simulation}: Fission-fusion dynamics, because that drives
migration and cooperation. We use a slightly reduced model of their decision
process.

\subsection{Interaction between culture and demographics}
\label{s:interaction}

Culture drives cooperation, cooperation pushes cultural assimilation. Following
\parencite{barcelo2014social,barcelo2015simulating}, but deviating from their
implementation (see \cref{s:geography} for some reasons why we can), we take this
as the core interation feedback loop.


\section{Parameter choices}
\label{s:parameters}
\section{Results}
\label{s:results}
\section{Discussion}
\label{s:discussion}
\section{Conclusions}
\label{s:conclusions}

\printbibliography
\end{document}

% Local Variables:
% TeX-engine: luatex
% TeX-command-extra-options: "-shell-escape"
% End:
